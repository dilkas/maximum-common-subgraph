\documentclass{article}

\usepackage[UKenglish]{babel}
\usepackage[UKenglish]{isodate}
\usepackage[backend=bibtex]{biblatex}

\bibliography{references.bib}

\author{Paulius Dilkas}
\title{Algorithm Selection for Maximum Common Subgraph}

\begin{document}
\maketitle

\section{Algorithms}
Describe how each algorithm works. Cite them.

\section{Features}
Features are based on the algorithm selection paper for the subgraph isomorphism
problem \cite{DBLP:conf/lion/KotthoffMS16}. Simple features:

\begin{itemize}
\item number of vertices,
\item number of edges,
\item density,
\item number of loops,
\item mean degree,
\item maximum degree,
\item whether every vertex has the same degree (?),
\item whether the graph is connected,
\item mean distance between all pairs of vertices,
\item maximum distance between all pairs of vertices,
\item proportion of vertex pairs that are at least 2, 3 and 4 apart (?).
\end{itemize}

Features that could be computed if we end up using a presolver:

\begin{itemize}
\item uniformity of the distribution of edges,
\item how many candidate pairs were removed,
\item proportion of candidate pairs removed over all pairs,
\item min values removed per variable,
\item max values removed per variable,
\item CPU time taken to compute all this.
\end{itemize}

\section{Selection Model}
We're using llama \cite{kotthoff_llama_2013}. Describe k-folding.

\printbibliography
\end{document}