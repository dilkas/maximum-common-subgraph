\documentclass{article}

\usepackage[UKenglish]{babel}
\usepackage[UKenglish]{isodate}
\usepackage[backend=bibtex]{biblatex}

\bibliography{references.bib}

\author{Paulius Dilkas}
\title{Algorithm Selection for Maximum Common Subgraph}

\begin{document}
\maketitle

\section{The Problem}
Labeled and unlabeled (undireced) maximum common induced subgraph. NB: we are
allowing loops. Update formulas accordingly.

\section{Algorithms}
Clique encoding \cite{DBLP:conf/cp/McCreeshNPS16}
$k\downarrow$ \cite{DBLP:conf/aaai/HoffmannMR17}
\textsc{McSplit} \cite{DBLP:conf/ijcai/McCreeshPT17}

(Somewhere) 1000 s limit, 512 GB limit (clique becomes impossible for some
instances), insert CPU specs.

\section{Problem Instances}

\subsection{ARG database}
MCS data is from \cite{DeSanto2003}\cite{foggia2001-2}.

\subsection{Benchmarks for the Subgraph Isomorphism Problem}
SIP instances are taken from \cite{solnon} (with the biochemical reaction
dataset excluded since we are not dealing with directed graphs).

\section{Features}
Features are based on the algorithm selection paper for the subgraph isomorphism
problem \cite{DBLP:conf/lion/KotthoffMS16}. Simple features:

\begin{itemize}
\item number of vertices,
\item number of edges,
\item density,
\item number of loops,
\item mean degree,
\item maximum degree,
\item standard deviation of degrees,
\item whether the graph is connected,
\item mean distance between all pairs of vertices,
\item maximum distance between all pairs of vertices,
\item proportion of vertex pairs that are at least 2, 3 and 4 apart (?),
\item number of labels,
\item number of distinct labels.
\end{itemize}

The last two were later rethought to be unnecessary and replaced by a boolean
feature ``labelled'' because if labelling is enabled, the number of labels is equal
to the number of vertices and the number of distinct labels is equal to 33\% of that.

Features that could be computed if we end up using a presolver:

\begin{itemize}
\item uniformity of the distribution of edges,
\item how many candidate pairs were removed,
\item proportion of candidate pairs removed over all pairs,
\item min values removed per variable,
\item max values removed per variable,
\item CPU time taken to compute all this.
\end{itemize}

Number (distinct or not) of labels includes the label of not having a label (so
it's always at least 1).

\section{Selection Model}
We're using \textsc{Llama} \cite{kotthoff_llama_2013}. Describe k-folding.

\printbibliography
\end{document}